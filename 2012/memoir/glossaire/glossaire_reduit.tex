
\newacronym{qos}{QoS}{Quality of Service}
\newacronym{tcp}{TCP}{Transmission-Control Protocol}
\newacronym{ip}{IP}{Internet Protocol}
\newacronym{udp}{UDP}{User Datagram Protocol}
\newacronym{osi}{OSI}{Open Systems Interconnection}
\newacronym{iso}{ISO}{International Standard Organization}

\newacronym{snmp}{SNMP}{Simple Network Management Protocol}
\newacronym{cmip}{CMIP}{Common Information Management Protocol}
\newacronym{lan}{LAN}{Local Area Network}
\newacronym{wan}{WAN}{Wide Area Network}
\newacronym{tmn}{TMN}{Telecommunication Management network}
\newacronym{ieee}{IEEE}{Institute of Electrical and Electronics Engineers}

\newglossaryentry{supervision}
{
name=Supervision,
description={
Superviser consiste à indiquer/modifier l'état d'un système informatique ou d'un hôte. La supervision  est donc la fonction qui permet de remonter/éditer les informations techniques relatives à un hôte ou a un ou des réseaux. En d'autres termes, la supervision de réseaux peut être définie comme l'utilisation de ressources réseaux (matériel et logiciel) adaptées dans le but d'obtenir des informations (en temps réel ou non) sur l'utilisation ou la condition des hôtes dans le réseau  afin d'assurer un niveau de service garanti, une bonne qualité et une répartition optimale et de ceux-ci.}
}

\newglossaryentry{typologie}
{
name=Typologie,
description={
La topologie d'un réseau peut se définir comme l'étude de la structure du réseau et des interconnexion entre hôtes. La typologie réseau peut être de type \emph{link layer topology}, \emph{network layer topology} ou \emph{Internet topology} et \emph{overlay topology}.}
}

\newglossaryentry{seeddb}
{
name=seedDB,
description={
L'étape du \emph{Seed Database} permet de fournir les informations importantes pour NAV pour pouvoir parcourir et découvrir le réseau. En effet, après installation et démarrage de NAV, NAV ne fait rien si les paramètres n'ont pas été fournis à la base.}
}

\newglossaryentry{agent}
{
name=Agent (SNMP),
description={Composant logiciel ou matériel qui gère les MIBs d'un matériel. Les agents répondent aux requêtes du Manager ou lui envoie un trap en cas d'alerte.}
}

\newglossaryentry{trap}
{
name=Trap,
description={Message d'alerte (problème qui nécessite une attention) envoyé par un agent au manager suite à un incident.}
}

\newglossaryentry{manager}
{
name=Manager,
description={Le manager est l'élément actif de la supervision. Il collecte les informations des hôtes en interrogeant les agents. Il reçoit aussi les traps provenant des agents. }
}

\newglossaryentry{standard}
{
name=Standard,
description={Un standard est une norme, un modèle à suivre. Dans le contexte des RFCs, un standard est le statut d'une proposition de norme, après les états de \emph{proposed} et \emph{draft}}.
}

\newglossaryentry{netkit}
{
name=NetKit,
description={Netkit permet de créer plusieurs  machines virtuelles sur le même hôte. Chaque machine virtuelle est reliée à un ou des domaines de collisions virtuels permettant ainsi les communications entre-elles. Chaque machine virtuelle peut jouer le rôle de PC, routeur ou switch.}
}

\newacronym{sgmp}{SGMP}{Simple Gateway Management Protocol}
\newacronym{asn}{ASN.1}{Abstract Syntax Notation .1}
\newacronym{ber}{BER}{Basic Encoding Rules}
\newacronym{mib}{MIB}{Management Information Base}
\newacronym{oid}{OID}{Object IDentifier}
\newacronym{nms}{NMS}{Network Management System}
\newacronym{smi}{SMI}{Structure of Management Information}
\newacronym{rfc}{RFC}{Request for Comments}
\newacronym{ietf}{IETF}{Internet Engineering Task Force}
\newacronym{itu}{ITU}{International Telecommunication Union}

\newacronym{sha}{SHA}{Secure Hash Algorithm}

\newacronym{pdu}{PDU}{Protocol Data Unit}
\newacronym{md5}{MD5}{Message Digest 5}
\newacronym{des}{DES}{Data Encryption Standard}
\newacronym{usm}{USM}{User-based Security Model}
\newacronym{vacm}{VACM}{View Access Control Model}

\newacronym{mac}{MAC}{Media Access Control}
\newacronym{icmp}{ICMP}{Internet Control Message Protocol}
\newacronym{egp}{EGP}{Exterior Gateway Protocol}
\newacronym{aes}{AES}{Advanced Encryption Standard}
\newacronym{popi}{POP}{Point Of Presence}
\newacronym{as}{AS}{Autonomous System}
\newacronym{dns}{DNS}{Domain Name System}

\newacronym{gpl}{GPL}{Global Public Licence}
\newacronym{nav}{NAV}{Network Administration Visualized}
\newacronym{noc}{NOC}{Network Operation Center}
\newacronym{acl}{ACL}{Access List Control}
\newacronym{iana}{IANA}{Internet Assigned Numbers Authority}
