\section{Les algorithmes}
Se basant sur les études citées dans la section précédente, nous proposons un algorithme pour initialiser la base de données conformément au \emph{minimum requirement} c'est-à-dire le \emph{Level 1: Minimum requirements}\footnote{\url{http://nav.uninett.no/seedessentials}} du \emph{seed database}.


L'algorithme proposé par Pandey and et. al. (2011) se déroule en étapes successives:

\begin{verbatim}
1. Take network information inputs
2. Device discovery
       a. Device discovery using next hop mechanism
       b. Device discovery using ARP cache entries
3. Device type discovery
4. Device grouping based on IP address
5. Connectivity discovery
       a. L2 to L2 connectivity
       b. L2 to L3 connectivity
       c. L3 to L3 connectivity
       d. L2 and L3 to end host connectivity
6. Subnet discovery and connectivity discovery in subnet
7. VLAN discovery and connectivity discovery in VLAN

\end{verbatim}



Un projet d'algorithme est aussi précisé sur la page du wiki\footnote{\url{http://metanav.uninett.no/devel:blueprints:autodiscovery-wizard}}.

La  première étape est la récupération des paramètres nécessaires pour la découverte des hôtes du réseau. Comme paramètres proposés nous avons:
\begin{itemize}
\item La \emph{communauté SNMP} pour l'accès en  lecture des MIBs.
\item Un préfixe ou une liste des préfixes (plage d'adresses IP de la découverte réseau).
\item L'adresse IP du routeur de départ.
\item Le nom de l'organisation auquel appartient les adresses IP (les équipements).
\end{itemize}
Une fois les paramètres donnés au script, les étapes suivantes sont proposées:


\begin{verbatim}
1. Retrieve all active arp records 
2. For each of these IP addresses
    a. If the IP address exists in NAV - skip
    b. Tf the address answers to SNMP : new device found!
    c. Get system.syslocation
    d. Set orgid to the supplied orgid
    e. Decide category
    f. Make new row to bulk file (format: roomid:ip:orgid:catid:ro )
\end{verbatim}

\paragraph{Notre algorithme est une mise en commun des quatre premières étapes proposées par Pandey and et. al. (2011) et de l'algorithme proposé pour la fonctionnalité de découverte des équipements du réseau. Nous ne prenons pas en compte la découverte de la connectivité (topologie)  et des VLAN. \texttt{ipdevpoll} s'occupe déjà de ces deux dernières fonctionnalités dans NAV. \\}

Notre algorithme de découverte du réseau se présente en six étapes suivantes:
\begin{verbatim}
1. Starting Point (Take network information inputs)
2. Device discovery
       a. Device discovery using next hop mechanism
       b. Device discovery using ARP cache entries
3. Device grouping based on IP address
4. Device type discovery
5. Create Bulk format
	   a. Set default roomid.
	   b. Set default orgid
	   c. Set category (type discovery)
	   d. Make new row to bulk file (format: roomid:ip:orgid:catid:ro )
6. Store in Db
\end{verbatim}
%a. If the IP address exists in NAV - skip
%b. Call bulk function to save in DB
%La dernière étape correspond essentiellement l'étape 2 de l'algorithme proposé par Vidar Faltinsen.


\subsection{Starting Point}
{\fontfamily{phv}\selectfont
\begin{center}
\begin{algorithm}
\caption{Starting Point}
\label{starting_point}

\begin{algorithmic}[1]
\Require $R\_IP \gets Starting ~ router~ IP~for ~ network~ discovery$
\Require $V\_SNMP \gets SNMP~ version$
\Require $C\_SNMP ~ community~ or~ security~ parameters~$
\Ensure $boolean \gets Router ~ SNMP~ availability~$

\Function{Starting\_Point}{$R\_IP,V\_SNMP,C\_SNMP$}
   \If{$R\_IP~ is~ manageable $}
    \State \Return $True$
   \Else
    \State \Return $False$
   \EndIf
    
\EndFunction

\end{algorithmic}

\end{algorithm}
\end{center}
}
Avec le mode de fonctionnement asymétrique de \emph{SNMP},   \emph{\gls*{agent}} et  \emph{\gls*{manager}} utilisent comme protocole de transport l' \emph{\gls*{udp}}.\\
Or \emph{UDP} (couche 4) se base sur le protocole  \emph{\gls*{ip}} (couche 3). Le critère de base de  fonctionnement de \emph{SNMP} est donc la présence dans le réseau d'équipement à même de supporter \emph{UDP} et par conséquent \emph{IP}. De manière plus simple, il faut que les équipements  membres du  \emph{\gls*{nms}} possèdent une adresse \emph{IP} valide.\\
La découverte d'un équipement du réseau repose donc sur ce critère de base: les équipements en communication doivent avoir chacun une adresse IP valide. L'adresse IP du \textbf{routeur} spécifié est considérée comme adresse source (algorithme \ref{starting_point}, page \pageref{starting_point}). Si le routeur ne supporte pas SNMP ou ne répond pas correctement à la requête SNMP (Community invalide), le processus de découverte s'arrête. Dans le cas contraire, le processus passe à l'étape suivante: la découverte des équipements connecté à ce routeur.\\


\subsection{Device Discovery}
Si l'étape du \emph{Staring Point} est concluant, le processus de découverte des hôtes du réseau peut commencer (algorithme \ref{device_discovery}, page \pageref{device_discovery}).

La découverte des hôtes du réseau se base sur deux groupes de \emph{MIB} du \emph{MIB-II}: \texttt{ipRouteTable} et \texttt{ipNetToMediaTable}.\\
\texttt{ipRouteTable} contient les variables relatives aux règles de routage du protocole \emph{IP}. Il permet donc de découvrir les adresses \emph{IP} des \textbf{routeurs} connectés avec le routeur courant. Dans cette table deux variables nous intéressent: \texttt{ipRouteNextHop} et \texttt{ipRouteType}. Le \texttt{ipRouteNextHop} indique l'adresse IP du \emph{Next Hop} (un routeur) si son \texttt{ipRouteType} est \emph{indirect}. \texttt{ipRouteTable} permet de découvrir \textbf{tous les routeurs} (hôtes de niveau 3) du réseau. En effet, en cas de découverte d'un routeur, son \emph{Next Hop} permet de découvrir d'autres routeurs du réseau et ainsi de suite. \textbf{Il est très improbable qu'un routeur ne supporte pas SNMP} (\emph{SNMP} s'appelait \emph{SGMP}).

La table \texttt{ipNetToMediaTable} contient les informations relatives à la couche 2. Elle permet donc de découvrir le couple adresse \emph{IP}/ adresse \emph{MAC} des hôtes du même réseau (réseau local).
\paragraph{}
La première étape de la découverte est constituée par la création du tableau vide \texttt{D[]}. Ce tableau contiendra les adresses IP de tous les équipements retrouvés dans le réseau. L'intérêt de ce tableau est double: avoir la liste des équipements du réseau et éviter à parcourir le même équipement plusieurs fois. Un routeur ayant par définition plusieurs interfaces.

Le premier routeur connu est le \emph{Starting Point}. Son adresse est alors ajoutée au tableau \texttt{D[]}.

La découverte du réseau commence par la recherche des routeurs connecté au \emph{Starting Point}.


{\fontfamily{phv}\selectfont
\begin{center}
\begin{algorithm}
\caption{Device Discovery}
\label{device_discovery}

\begin{algorithmic}[1]
\Ensure $D[] ~ all~ network~ device~ IP~array$
\Function{Device\_Discovery}{}
 \State $D[] \gets all~ network~ visited~ routers~ and~ device~ array,~ initially~ empty$
 \State $D[].push(Starting\_Point\_IP) \gets add~ Starting~ router~ IP~ to~ R[]~ stack $
   \ForAll{$D[n]$}
    \Function{getIndirectNextHop}{}
     \If{$(getRequest(D[n])==True)$} $\gets ipRouteNextHop$
      \State $N\_H[] \gets ~next~ hop~ set~ for~ R[n] \gets ipRouteNextHop~ if~ ipRouteType~ is~ indirect$
      \ForAll{$N\_H[m]$}
       \If{$(D[]~ contains~N\_H[m])$} 
        \State $continue$
       \Else
        \State $D[].add(N\_H[m])$
        \State $Device\_Grouping(D[])$
       \EndIf      
       
      \EndFor
      \EndIf
    \EndFunction
   \EndFor
   
   \Function{getLocalNetAddress}{}
   \ForAll{$D[i]$}
    \If{$(getRequest(D[i])==True)$} $\gets ipNetToMediaNetAddress $
     \State $N\_D[] \gets nettomediatable~ for~ D[i] $
     \ForAll{$N\_D[j]$}
      \If{$D[]~contains~N\_D[j] $}
       \State $continue$
      \Else
       \State $D[].add(N\_D[j])$
       \State $Device\_Grouping(D[])$
      \EndIf
     \EndFor
    \EndIf
   \EndFor
  \EndFunction
    
\EndFunction

\end{algorithmic}

\end{algorithm}
\end{center}
}

\subsubsection{Device discovery using next hop mechanism}
Pour chaque routeur dans le tableau \texttt{D[]} (à la première itération, nous n'avons que le \emph{Sarting Point}), nous cherchons les valeurs du \emph{MIBs} \texttt{ipRouteNextHop} de type \emph{indirect}. Si on en trouve, on crée localement le tableau \texttt{N\_H[]} contenant les adresses IP de tous les routeurs connectés au routeur courant. Ensuite, nous vérifions pour tous les routeurs contenus dans \texttt{N\_H[]}, la présence de leurs adresses IP dans \texttt{D[]}. Si le routeur n'est pas encore connu, son adresse est ajouté à \texttt{D[]}. \texttt{D[]} contient donc de plus en plus d'éléments permettant à la boucle de tourner. 

Au cas où le routeur n'aurait pas de table \texttt{ipRouteNextHop}, alors ce n'est pas un routeur mais un tout autre équipement qui sera découvert par l'étape suivante.
 
\subsubsection{Device discovery using ARP cache entries}
Lorsqu'il n'y aura plus de routeur non parcourus,  \texttt{D[]} contiendra la liste de tous les routeurs visités du réseau. Pour chacun de ces routeurs, nous cherchons dans le cache ARP (\texttt{ipNetToMediaTable}), les adresses IP des hôtes de son réseau local que nous stockons dans le tableau \texttt{N\_D[]}. Pour chaque adresse trouvée, nous vérifions l'unicité dans \texttt{D[]} pour éviter les doublons: le routeur est d'une part connecté à un autre routeur, d'autre par connecté à un autre réseau local. Dans le cache de ce routeur, nous aurons les adresses IP  non seulement des hôtes du réseau local mais aussi celui des autres routeurs auquel il est connecté. La vérification dans le tableau \texttt{D[]} permet d'éliminer les adresses des autres routeurs connectés à celui-ci. Si nous avons une nouvelle adresse, elle est ajoutée à \texttt{D[]}

\paragraph{Le problème de l'ARP\\}
L' \emph{Address Resolution Protocol (ARP)} assure l'établissement dans le cache de l'interface réseau , d'une table de correspondance entre adresse \emph{MAC} et adresse \emph{IP} des autres hôtes du même réseau local. Ce protocole n'est pas sécurisé puisqu'il est sujet au \emph{cache poisonning}: il est possible de faire aussi bien de l'usurpation d'adresse IP que d'adresse MAC. Dans un réseau commuté, il suffit de faire croire au switch que nous sommes l'hôte possédant l'adresse que nous souhaitons écouter, et renvoyer après le trafic au véritable détenteur de l'adresse usurpée. De ce fait, tout le trafic passe par notre hôte.\\
Aussi la découverte d'hôtes grâce à l'\emph{ARP} peut échouer à cause du TTL du caching au niveau des interfaces. Après un certain temps d'inactivité d'un hôte sur le réseau local, les informations relatives à cet hôte sont effacées du cache des autres interfaces du même réseau. Un hôte peut bien être dans le réseau local mais parce qu'il est éteint par exemple, la découverte par \emph{ARP} va échouer.\\

Une solution est d'utiliser \emph{ICMP (Internet Control Message Protocol)} à la place de l'\emph{ARP} dans le réseau local. Au cas où le contenu de la table \texttt{ipNetToMediaTable} serait vide, nous pourrions envoyer des requêtes \emph{ICMP} \emph{type 8 (Echo)} c'est-à-dire des \emph{pings} aux adresses IP du réseau local. Ces \emph{pings} réguliers vont permettre de rafraichir le cache des interfaces des hôtes du réseau local. \'A noter que même si un hôte ne répond pas avec une réponse de \emph{type 0 (Echo response)}, les informations relatives au couple adresse \emph{IP / MAC} sont renvoyées. Mais dans le cas de cette étude, nous nous limiterons au contenu du \texttt{ipNetToMediaTable}.

\subsection{Device grouping}
\begin{center}
\begin{algorithm}
\caption{Device Grouping}
\label{device_grouping}
\begin{algorithmic}[1]
\Require $D[] ~ all~ network~ device~ IP~array$
\Ensure $D\_G[]:~ all~ network~ ~grouping~device~ IP~array$

\Function{Device\_Grouping}{}
 \ForAll{$D[i]$}
  \State $IP\_T[] \gets all ~IP~ in~ ipAddrTable~ related~ to~ curent~ device~ \gets ipAdEntAddr$
   
   \ForAll{$IP\_T[i]$}
    \If{$IP\_T[i]~equal~D[i]$}
     \State $continue$
    \Else
     \If{$D[]~contains~IP\_T[i]$}
      \State $D[].pop(IP\_T[i])$
     \EndIf
    \EndIf
    
   \EndFor
 \EndFor

   
    
\EndFunction

\end{algorithmic}
\end{algorithm}
\end{center}
Cette étape permet d'éviter les \emph{synonymes} dans le réseau. Un même hôte pouvant avoir plusieurs adresses IP (sur plusieurs interfaces). Si par exemple un hôte était doublement connecté sur le réseau (physique et sans fil), nous aurions dans \texttt{D[]}  deux adresses IP différentes signifiant deux hôtes différents au lieu d'un seul. De même vu le modèle de réseau virtuel de test (figure \ref{fig:snmp_lab}, page \pageref{fig:snmp_lab}), on aurait dans \texttt{D[]} un nombre supérieur à six (nous n'avons que six routeurs dans ce réseau) d'adresses IP.\\
Une solution est de vérifier pour chaque adresse IP d'équipement ajouté dans \texttt{D[]}, si ce dernier ne possèdent pas  en plus de son adresse connue, d'autres adresse enregistrées dans \texttt{D[]}. La table \texttt{ipAddrTable} contient pour chaque interface les informations sur l'adresse IP et l'adresse MAC. Pour chaque  équipement du réseau dans le tableau \texttt{D[]}, nous récupérons sa table \texttt{ipAddrTable}. Pour chaque adresse IP de la table hormis l'adresse avec laquelle l'équipement est parcouru, nous vérifions la présence dans \texttt{D[]}. Si l'adresse IP est présent, nous l'enlevons du tableau des équipements \texttt{D[]}.
\paragraph{L'étape du device grouping s'opère à chaque ajout d'un nouvel équipement dans \texttt{D[]}. C'est une étape exécutée aussi bien au niveau du \emph{Device discovery using next hop mechanism} que du \emph{Device discovery using ARP cache entries}.}


\subsection{Device type discovery}
Avec le tableau \texttt{D[]} , il devient possible de découvrir les \emph{types} d'hôte. La variable \texttt{sysService} de \emph{SNMP} nous permet d'identifier certains types d'hôtes en fonction des services offerts.\\
\texttt{sysService} est une variable qui indique le type de service offert par l'équipement. Elle est représentée par la somme des bits correspondant aux couches OSI auxquelles l'équipement fournis des services. Le tableau \ref{sysServices_mib} (\nameref{sysServices_mib}) montre les valeurs en termes de position de bit pour les couches du modèle TCP/IP (RFC 1213).
\begin{table}[h]
\begin{center}
 \begin{tabular}{|l|l|}
 \hline
 \textbf{Couche} & \textbf{Service (matériel)} \\\hline
 \texttt{1} & physique (repeteur) \\\hline
 \texttt{2} & liaison (switch non manageable) \\\hline
 \texttt{3} & réseau (switch manageable, routeur) \\\hline
 \texttt{4} & transport (hôtes) \\\hline
 \texttt{7} & application (mail, http, etc) \\\hline
 \end{tabular}
\caption{Valeur de la variable sysServices du  MIB-II (RFC 1213)} \label{sysServices_mib}
\end{center}
\end{table}
Le résultat du \texttt{sysService} détermine les autres \emph{MIB}s à prendre en compte pour déterminer les types des hôtes. Le \emph{RCF 1242} donne les \emph{définitions} de quelques hôtes utilisées dans un réseau informatique.

\begin{center}
\begin{algorithm}
\caption{Device Type}
\label{device_type}
\begin{algorithmic}[1]
\Require $D[] ~ all~ network~ device~ IP~array$
\Ensure $D\_Type:~ local~ network~ devices~ type$

\Function{Device\_Type}{}
 \ForAll{$D[i]$}
  \If{$sysServices(D[i])$}
  
   	\If{$sysServices(D[i])~ equal ~ 78 $}
   	 \State $D[i].append(GW)$
   	 
   	\ElsIf{$sysServices(D[i])~ equal ~ 72 $}
   	
   	 \If{$ipForwarding(D[i])~ equal ~ 1$}
   	 \State $D[i].append(GW)$
   	 \Else
   	  \State $D[i].append(SRV)$
   	 \EndIf
 
   	\Else
   	 \State $D[i].append(OTHER)$
   	\EndIf
     	 
   	\ElsIf{$dot1dBridge(D[i])$}
   	 \State $D[i].append(SW)$
   	 
   \Else
    \State $D[i].append(OTHER)$  
   \EndIf
 \EndFor
   
    
\EndFunction

\end{algorithmic}
\end{algorithm}
\end{center}

\paragraph{Router\\}
Le premier type d'équipement à identifier est le routeur.\\
Un routeur est un équipement qui transfert les paquets. La variable \texttt{ipForwarding} (RFC 4292) sera à 1. Aussi la variable \texttt{sysService} sera \texttt{1001110}. Le second bit (de poids faible), le troisième, le quatrième et le septième sont à 1; ce qui donne en base 10, 78. Un routeur offre les services des couches 2 (liaison), couche 3 (adressage), couche 4 (transport) et couche 7 (application). Un hôte qui sert de passerelle dans un réseau sera considéré comme un routeur. Il en est de même pour un routeur-switch.

\paragraph{Host\\}
Les Hosts sont des équipements qui ne transfère pas de paquet mais qui dispose de table de routage. Ces hôtes peuvent être des PC, des imprimantes, des serveurs, etc. Pour plus de précision sur les types d'hôtes, nous pouvons utiliser les \emph{Product specific MIB} tels \emph{Printer MIB}, \emph{WWW-MIB}, \emph{APACHE-MIB}, etc. Mais dans le cadre de notre étude, nous nous limiterons à la valeur de \texttt{sysService} qui est 72\footnote{RCF 1213, p 15} (1001000) pour les Hosts. Les Hosts offrent donc des services de niveau 4 (transport) et niveau 7 (application)

\paragraph{Bridges\\}
Si l'équipement n'est ni un routeur, ni un simple hôte, il pourrait être un switch.\\
Un pont ou switch de niveau 2, est un équipement transparent (au niveau 3) dans le réseau. Il ne transfert pas de paquet au niveau IP et ne dispose pas de table de routage. Le transfert de trame se fait au niveau liaison grâce au maintient d'une                                                 table d'adresses des cartes réseaux qu'il voit sur ses ports : l'\emph{Address Forwading Table (AFT)} (RFC 4188).\\
La détection des switchs pourrait se baser sur les \emph{Bridge-MIB}. Si un équipement dispose du \emph{MIB} \texttt{dot1dBridge} alors nous pouvons le considérer comme un switch.


Dans le contexte de NAV, un équipement est soit un GW (routeur), GSW (routeur-switch), SW (switch), WLAN (AP), EDGE (switch en contact direct avec les ordinateurs. NAV ne collecte pas de données depuis ces switchs), SRV (serveurs) et OTHER (autres type d'équipements). Dans le cas ce cette étude, nous ne pouvons que identifier grossièrement les \texttt{GW}, les \texttt{SW} et les \texttt{OTHER}. \emph{SNMP} ne permet pas directement de faire la distinction entre un PC (\texttt{OTHER}) et un serveur (\texttt{SRV}). Les deux pouvant offrir les mêmes services.Tout Host sera considéré comme \texttt{OTHER}. Il s'en va dire que même après l'identification de certains hôtes avec \emph{SNMP}, la probabilité d'erreur est très proche de un et une identification manuelle est obligatoire: un routeur-switch pouvant être considéré comme un routeur (table de routage).




\subsection{Create Bulk format}
L'initialisation de la base se fera conformément au \emph{Level 1: Minimum requirements}\footnote{\url{http://nav.uninett.no/seedessentials}} du \emph{seed database}. Ainsi des valeurs par défaut existe déjà pour le \emph{roomid}: \textbf{\texttt{myroom}} et l'\emph{orgid}: \textbf{\texttt{myorg}}. Nous avons les adresses IP et leur type dans le tableau \texttt{D[]}. Pour chaque couple (adresse IP, type), il ne nous reste plus qu'à ajouter les valeurs communes telles le \emph{roomid} et l'\emph{orgid}, puis ajouter le \texttt{community} pour tous les équipements sauf les serveurs. 

Pour chaque équipement, une ligne est ajouté dans un fichier texte, contenant tous les équipements du réseau.

\subsection{Store in Db}
Pour le stockage dans la base du données, nous allons utiliser la méthode d'import en masse déjà disponible dans le NAV. En effet l'interface graphique du NAV permet de faire des importations à partir d'un fichier respectant le format adéquat. Cette importation en masse se fait à partir du menu \texttt{Home > Seed DB > IP Devices} puis \texttt{Bulk import}.

L'étape précédente, celle du \textbf{Create Bulk format} nous a déjà permis d'obtenir un fichier suivant le format adéquat. Il ne nous reste plus qu'à l'importer dans la base du NAV.




