\section{Initialisation automatique de la base de données grâce à SNMP}
\monBloc{L'initialisation automatique de la base de données grâce à SNMP constitue notre apport. En effet, avec le modèle actuel proposé par NAV, il faudra actuellement soit ajouter un à un les équipements du réseau, soit remplir un formulaire où chaque ligne correspond à chaque équipement au niveau de l'interface d'administration. Cette tâche s'avère très fastidieuse d'où la nécessité de l'automatiser.}

C'est dans cette perspective que la fonctionnalité \emph{Autodiscovery wizard for NAV} a été prévu sur la feuille de route des futurs développements\footnote{\url{https://blueprints.launchpad.net/nav/+spec/autodiscovery-wizard}}


Nous proposons d'écrire un script permettant de découvrir tous les hôtes du réseau supportant SNMP à partir d'un point de départ. Une fois ces hôtes découvert, il faudra les mettre sous un format acceptable par NAV et les insérer dans la base de données.


\subsection{Études relatives à la découverte de la typologie réseau avec SNMP}
De nombreux études ont porté sur la découverte de la topologie physique et logique des réseaux informatiques. Nous ne citerons ici que les quatre dernières études en la matière:
\begin{enumerate}
\item Pandey S, Choi M, Lee S, Hong J. \footnote{
Suman Pandey, Mi-Jung Choi, Sung-Joo Lee, James W. Hong	Dept. of Computer Science and Engineering, POSTECH, Korea}.\emph{IP Network Topology Discovery Using SNMP}. International Conference on Information Networking (ICOIN'09). 23 33-37 2009 %\cite{AT1}
\item Yang Xiao. \emph{Physical Path Tracing for IP Traffic using SNMP}. BBC Research White Paper WHP 188. 2010 %\cite{AT2}
\item Pandey S, Choi M, Won Y, Hong J. \emph{SNMP-based enterprise IP network topology discovery. International Journal of Network Management} 21 (3) 169-184. 2011 %\cite{AT3}
\item Musa Balta, Ibrahim Özçelik. \emph{The Discovery of Enterprise Network Topology Created in a Virtual Environment with SNMPv3}. The Online Journal of Science and Technology (TOJSAT) Volume 2, Issue 2, 64-70, Avril 2012. %\cite{AT4}
\end{enumerate}
Toutes les trois dernières études se basent sur les résultats de la première étude de la liste ci-dessus: l'étude sur \emph{IP Network Topology Discovery Using SNMP}. 

De ces études, la typologie réseau découverte est de type physique et logique soit le \emph{link layer topology} et le \emph{router level} du \emph{Internet topology} d'une organisation ou d'un AS. C'est donc une découverte de réseau interne à une organisation ou un AS.

La typologie d'un réseau désigne l'organisation des éléments (équipements et liens) du réseau et les interactions physiques et logiques entre ces divers éléments.

La typologie logique se réfère à la structuration ou division logique du réseau en sous réseaux. %Suivant la logique de division considérée, on peut avoir plusieurs typologies logiques pour un même réseau (cas des switchs L2, L3, L4, L5 ou L7).

La typologie physique désigne les communications ou connexions visibles entre hôtes sur des ports grâce à des liens (de transmission).

Dans le cadre de l'automatisation de l'initialisation de la base de NAV, nous n'implémenterons que certains aspects des résultats de ces études. En effet, la fonctionnalité offerte par le \texttt{ipdevpoll} permettent déjà de faire de la mise à jour des informations dans la base. Notre apport permettra de faire des insertions en masse des informations minimales sur chaque équipent découvert dans le réseau.

