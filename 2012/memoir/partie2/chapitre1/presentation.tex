\section{Présentation rapide}
Le \textbf{Network Administration Visualized} est une suite logicielle de pointe pour superviser les réseaux informatiques de grande taille.

\texttt{NAV} présente les caractéristiques suivantes:

\begin{itemize}
\item Originaire de la Norvège: \emph{NTNU (Norvegian University of Science and Technology)}.
\item Licence GPL v2. 
\item Sous la responsabilité du \emph{UNINETT (the Norvegian research network)}. 
\item Utilisation par de nombreuses universités dans le monde. 
\item Programmation en Python.
\end{itemize}

\paragraph{Intérêt du NAV\\}
\texttt{NAV} est une collection intégrée de solutions libres ayant déjà fait leurs preuves. Ce qui fait de \texttt{NAV}, la seule solution distribuée offrant de nombreuses fonctionnalités.


\paragraph{}
Comme fonctionnalités majeures du \texttt{NAV}, nous avons:
\begin{itemize}
\item \emph{Geographical Map}: typologie sur une carte \emph{OpenStreetMap}.
\item \emph{Network Weather Map}: état d'utilisation des liens (topologie).
\item \emph{Machine Tracker}: historique du couple adresse IP et adresse MAC pour chaque équipement.
\item \emph{Mac Watch}: surveillance des adresses MAC du réseau. 
\item \emph{Layer 2 trace}: trace du parcours d'un paquet d'une source à une destination.
\item \emph{Arnold}: blocage des ports des routeurs et switchs en les mettant en quarantaine.
\item \emph{Maintenance Tasks}: Exclusion d'un équipement du mécanisme de supervision sans enlever l'équipement du réseau.
\item \emph{Radius Accounting}: Suivi à la trace des utilisateurs nomades.
\end{itemize} 



\paragraph{}
Plus d'informations sont disponibles sur le wiki\footnote{\url{http://nav.uninett.no/navtechdoc}} du projet.